This translation of Sappho Fragment \#31 was not necessarily the most difficult translation that I have conducted, but it was by far the most enjoyable. Despite having the unfamiliar Aeolian dialect presenting many perculiarities in terms of declensions and obscure conjugations, the original text which was presented was interesting, and offered the translator a great deal of choice in how to convey the sense of emotion that the author originally wrought. As there are many great, literalist translations available in the literature, I have aimed to pursue a more poetic translation, one that will perserve the fluidity and emotion behind the original text, as well as offering a more rough interlinear translation.

Unlike my previous Apollodorus translation, I have opted to not include the comprehensive 'gloss tables' which extended my Apollodorus translation to more than 25 pages. With my current skill level in Greek, I no longer feel it is nessesary to gloss every single pronoun and conjunction. Indeed, when I started embarking on this translation, I used the Apollodorus template, but quickly found the act of filling out glosses to be menial and exhausting, and contributing very little to my understanding of the text.

Overall, I hope my translation of the famous Sappho Fragment \#31 will be an worthy one. For this poem speaks of love and attraction in a way that every human being can relate to, be it by the marble pillars of the Acropolis at Lesbos, or the humble lockers at a Highschool hall.
\clearpage

\begin{multicols}{2}
\section*{Original Aeolic Greek}
\begin{verse}
  φαίνεταί μοι κῆνος ἴσος θέοισιν \\
  ἔμμεν᾽ ὤνηρ, ὄττις ἐνάντιός τοι \\
  ἰσδάνει καὶ πλάσιον ἆδυ φωνεί-  \\
  σας ὐπακούει \\
  \medskip

  καὶ γελαίσας ἰμέροεν, τό μ᾽ ἦ μὰν \\
  καρδίαν ἐν στήθεσιν ἐπτόαισεν· \\
  ὠς γὰρ ἔς σ᾽ ἴδω βρόχε᾽, ὤς με φώναι--- \\
  σ᾽ οὐδ᾽ ἒν ἔτ᾽ εἴκει, \\
  \medskip

  ἀλλ᾽ ἄκαν μὲν γλῶσσα \emph{ἔαγε}, λέπτον \\
  δ᾽ αὔτικα χρῶι πῦρ ὐπαδεδρόμηκεν, \\
  ὀππάτεσσι δ᾽ οὐδ᾽ ἒν ὄρημμ᾽, ἐπιρρόμ--- \\
  βεισι δ᾽ ἄκουαι, \\
  \medskip

  κάδ δέ μ' ἴδρως φῦχρος ἔχει, τρóμος δὲ \\
  παῖσαν ἄγρει, χλωροτέρα δὲ ποίας \\
  ἔμμι, τεθνάκην δ᾽ ὀλίγω ᾽πιδεύης \\
  φαίνομ᾽ ἔμ᾽ αὔται· \\
  \medskip

  ἀλλὰ πὰν τόλματον ἐπεὶ \emph{καὶ πένητα}
\end{verse}
\columnbreak
\section*{Shen's Translation}
\begin{verse}
  He appears to me like the very Gods, \\
  the man who faces me. \\
  Sitting by my side, listening, \\
  to my sweet-spoken words. \\
  \medskip

  His lovely laugher, having me truly \\
  heart aflutter in my breast. \\
  Even a brief glance, \\
  makes my voice give away. \\
  \medskip

  Like my tongue is subtly broken, \\
  yet on the other hand, my skin on fire \\
  despite the eyes shying away, \\
  a whirlwind in my ears. \\
  \medskip

  For I tremble with anxiety, \\
  yellow like grass I am, \\
  almost as if I'm dead, \\
  it seems to me here. \\
  \medskip

  But yet, all this I must endure,

\end{verse}
\end{multicols}
\clearpage
\section*{Interlinear Translation}
This is a very literal, rough translation which aims to preserve the Greek syntax as much as possible.

\subsection*{Stanza 1}
\subsubsection*{φαίνεταί μοι κῆνος ἴσος θέοισιν}
He appears to me that same to the Gods

\subsubsection*{ἔμμεν᾽ ὤνηρ, ὄττις ἐνάντιός τοι}
To be the man whoever opposite facing you

\subsubsection*{ἰσδάνει καὶ πλάσιον ἆδυ φωνεί-}
And he sits close to [you], [you] speaking sweetly

\subsubsection*{σας ὐπακούει}
Listening to [you]

\subsection*{Stanza 2}
\subsubsection*{καὶ γελαίσας ἰμέροεν, τό μ᾽ ἦ μὰν}
And laugh[ing] lovely, this me in truth verily.

\subsubsection*{καρδίαν ἐν στήθεσιν ἐπτόαισεν·}
Heart in breast to flutter.

\subsubsection*{ὠς γὰρ ἔς σ᾽ ἴδω βρόχε᾽, ὤς με φώναι---}
For here into I see short [time], like my voice

\subsubsection*{σ᾽ οὐδ᾽ ἒν ἔτ᾽ εἴκει,}

but not yet it give away

\subsection*{Stanza 3}
\subsubsection*{ἀλλ᾽ ἄκαν μὲν γλῶσσα \emph{ἔαγε}, λέπτον}
But softly [on one hand] [her/my] tongue to be broken subtly

\subsubsection*{δ᾽ αὔτικα χρῶι πῦρ ὐπαδεδρόμηκεν,}
[on the other hand] at once [her/my] skin fire overrun

\subsubsection*{ὀππάτεσσι δ᾽ οὐδ᾽ ἒν ὄρημμ᾽, ἐπιρρόμ---}
Eye for not into see[ing](?), to make a buzzing noise [in]

\subsubsection*{βεισι δ' ἄκουαι}
[her/my] sense of hearing

\subsection*{Stanza 4}
\subsubsection*{κάδ δέ μ' ἴδρως φῦχρος ἔχει, τρóμος δὲ}
For into my sweat cold [it] has, trembling.

\subsubsection*{παῖσαν ἄγρει, χλωροτέρα δὲ ποίας}
Wholly [he/she/it] takes, greenish-yellow for grass

\subsubsection*{ἔμμι, τεθνάκην δ᾽ ὀλίγω ᾽πιδεύης}
to be, [to] die (?) for small in need of

\subsubsection*{φαίνομ᾽ ἔμ᾽ αὔται·}
seem to me here.

\subsection*{Stanza 5}
\subsubsection*{ἀλλὰ πὰν τόλματον ἐπεὶ \emph{καὶ πένητα}}
But all to be ventured since and poor
